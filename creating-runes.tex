\section{How Runes Work In-Universe}
Spellbinding runes imbue small objects with magical power. When objects with these runes carved into them are spread out around an area, they can be used to cast the spell both more quickly and over a wider and more precise area than normal. In exchange, a spellbinder must prepare the runes ahead of time and find a way to get the runes into the correct position to cast spells.

Each spell a spellbinder knows is associated with a type of rune. When an object has a rune inscribed on it, it can be used to cast the corresponding spell, but no other spells. Objects can only have one type of rune inscribed on it at a time.

When marking an object with a rune, the spellbinder cannot just inscribe the same symbol thoughtlessly. The rune must be adjusted to bind correctly depending on the material and precise shape of the object it is inscribed in. Simple spells can be marked easily, as the spell will still work well even if there are some errors in the rune, but more complicated spells have less room for error and require more complicated markings. As a result, the amount of time it takes to prepare runes varies from spell to spell.

Runes are most frequently carved into wooden stakes, but different materials can be used for the stakes, and some spellbinders also learn to inscribe runes into small orbs.

\section{Creating an Individual Rune Stake}
As long as a spellbinder has access to unmarked stakes or orbs, they can carve or draw runes onto it to prepare it for use in a spell. Spellbinders can buy stakes from stores, and they also start with woodworking tools they can use to cut branches off trees and carve stakes.

While the spellbinder is in downtime, they can take an unmarked stake and mark it with runes for one of their spells. Inscribing the runes does \textit{not} require performing a spell ritual, using spell components, or using a spell slot. However, inscribing does take time. Look up the time it takes a traditional spellcaster to cast the spell being inscribed; that time is how long it takes to inscribe a single stake with that spell's runes.\sidenote{If a spell requires 1 action to cast, it takes 6 seconds to inscribe its rune. If a spell can be cast instantaneously, it takes 1 second to inscribe its rune.} (Later, actually casting the spell takes one action, regardless of how long the spell takes to cast traditionally.)

In addition to inscribing runes in downtime, spellbinders can inscribe runes while traveling as long as they are doing nothing else requiring their attention or the use of their hands while traveling. Inscribing entails using a knife or pen to carve into small objects, so there's no reason it should be difficult to perform while walking on foot.

\section{Creating Multiple Rune Stakes}

To create multiple rune stakes, simply perform the inscribing process on each stake one a time. You can determine how much time your spell caster has to spend inscribing spells, then decide on which spells to carve, so long as you have time. If something interrupts the process, you must decide which spells you finished inscribing beforehand and leave the rest unfinished.

\section{Limitations on Carrying Prepared Stakes}

In order to use prepared rune stakes quickly, spellbinders have to keep their prepared runes organized. This prevents them from keeping too many prepared rune stakes on them at once. As they gain levels, they become more comfortable using their systems and can carry more rune stakes at once.

Table \ref{table:class-spellbinding-limits} in Chapter \ref{ch:class-description} shows how many prepared rune stakes a spellbinder can carry at once.