\section{Introduction}

\section{Class Features}

As a Spellbinder, you gain the following class features:

\subsection{Hit Points}
\paragraph{Hit Dice:} $1d8$ per spellbinder level
\paragraph{Hit Points at 1st Level:} $8 +\text{your Constitution modifier}$
\paragraph{Hit Points at Higher Levels:} $1d8$ (or $5$) $+$ your Constitution modifier per spellbinder level after 1st

\subsection{Proficiencies}
\paragraph{Armor:} Light armor, shields
\paragraph{Weapons:} Simple melee weapons
\paragraph{Tools:} Carpenter's tools
\paragraph{Saving Throws:} Intelligence, Dexterity
\paragraph{Skills:} Acrobatics, Arcana, Athletics, Stealth, and one other Intelligence- or Dexterity-based skill 

\subsection{Equipment}
You start with the following equipment, in addition to the equipment granted by your background:

\begin{itemize}
    \item Any simple melee weapon
    \item A woodcutting axe\sidenote{The woodcutting axe is weaker and heavier than a regular axe for combat. It deals $1d4$ slashing damage, weighs $5$ lbs., and cannot be thrown as a ranged weapon.}
    \item A woodcarving knife\sidenote{The woodcarving knife is equivalent to a dagger when used in combat, except that it cannot be thrown as a ranged weapon.}
    \item An explorer's pack
\end{itemize}

\subsection{Spellcasting}
A spellbinder uses the spellbinding casting system. Spellbinding is described in chapters 2, 3, and 4, so it will not be addressed in detail here. Briefly, spellbinding prepares sticks with runes in them that can be placed around a battlefield to create large areas affected by strong spells.

Table \ref{table:class-spell-slots} shows how many spell slots and known spells a spellbinder has acccess to at each level.

Table \ref{table:class-spellbinding-limits} shows how many rune stakes a spellbinder can carry at once and how close rune stakes have to be to each other at each level.

\begin{table}
\center
\begin{tabular}{r | r | r r r r r r r r r r}
\toprule
\multicolumn{2}{c}{} & \multicolumn{10}{c}{Spell Slots per Spell Level} \\
Level & Proficiency Bonus & Cantrips Known & 1st & 2nd & 3rd & 4th & 5th & 6th & 7th & 8th & 9th \\
\midrule
1st  & $+2$ & $3$ & $2$ & - & - & - & - & - & - & - & - \\
2nd  & $+2$ & $3$ & $3$ & - & - & - & - & - & - & - & - \\
3rd  & $+2$ & $3$ & $4$ & $2$ & - & - & - & - & - & - & - \\
4th  & $+2$ & $4$ & $4$ & $3$ & - & - & - & - & - & - & - \\
5th  & $+3$ & $4$ & $4$ & $3$ & $3$ & - & - & - & - & - & - \\
6th  & $+3$ & $4$ & $4$ & $3$ & $3$ & - & - & - & - & - & - \\
7th  & $+3$ & $4$ & $4$ & $3$ & $3$ & $1$ & - & - & - & - & - \\
8th  & $+3$ & $4$ & $4$ & $3$ & $3$ & $2$ & - & - & - & - & - \\
9th  & $+4$ & $4$ & $4$ & $3$ & $3$ & $3$ & $1$ & - & - & - & - \\
10th & $+4$ & $5$ & $4$ & $3$ & $3$ & $3$ & $2$ & - & - & - & - \\
11th & $+4$ & $5$ & $4$ & $3$ & $3$ & $3$ & $2$ & $1$ & - & - & - \\
12th & $+4$ & $5$ & $4$ & $3$ & $3$ & $3$ & $2$ & $1$ & - & - & - \\
13th & $+5$ & $5$ & $4$ & $3$ & $3$ & $3$ & $2$ & $1$ & $1$ & - & - \\
14th & $+5$ & $5$ & $4$ & $3$ & $3$ & $3$ & $2$ & $1$ & $1$ & - & - \\
15th & $+5$ & $5$ & $4$ & $3$ & $3$ & $3$ & $2$ & $1$ & $1$ & $1$ & - \\
16th & $+5$ & $5$ & $4$ & $3$ & $3$ & $3$ & $2$ & $1$ & $1$ & $1$ & - \\
17th & $+6$ & $5$ & $4$ & $3$ & $3$ & $3$ & $2$ & $1$ & $1$ & $1$ & $1$ \\
18th & $+6$ & $5$ & $4$ & $3$ & $3$ & $3$ & $3$ & $1$ & $1$ & $1$ & $1$ \\
19th & $+6$ & $5$ & $4$ & $3$ & $3$ & $3$ & $3$ & $2$ & $1$ & $1$ & $1$ \\
20th & $+6$ & $5$ & $4$ & $3$ & $3$ & $3$ & $3$ & $2$ & $2$ & $1$ & $1$ \\
\bottomrule
\end{tabular}
\label{table:class-spell-slots}
\caption{Spell slots and spells known at each level.}
\end{table}

\begin{table}
\center
\begin{tabular}{r | r | r r }
\toprule
\multicolumn{2}{c}{} & \multicolumn{2}{c}{Maximum Distance Between Rune Stakes} \\
Level & Rune Stake Carrying Capacity & For Tank Class Option & For Ranged Class Option \\
\midrule
1st  & $25$ & $30$ feet & $30$ feet \\
2nd  & $30$ & $35$ feet & $35$ feet \\
3rd  & $35$ & $40$ feet & $45$ feet \\
4th  & $40$ & $45$ feet & $60$ feet \\
5th  & $45$ & $50$ feet & $80$ feet \\
6th  & $50$ & $55$ feet & $105$ feet \\
7th  & $55$ & $60$ feet & $135$ feet \\
8th  & $60$ & $65$ feet & $170$ feet \\
9th  & $65$ & $70$ feet & $210$ feet \\
10th & $70$ & $75$ feet & $255$ feet \\
11th & $75$ & $80$ feet & $305$ feet \\
12th & $80$ & $85$ feet & $360$ feet \\
13th & $85$ & $90$ feet & $420$ feet \\
14th & $90$ & $95$ feet & $485$ feet \\
15th & $95$ & $100$ feet & $555$ feet \\
16th & $100$ & $110$ feet & $630$ feet \\
17th & $110$ & $125$ feet & $710$ feet \\
18th & $125$ & $150$ feet & $795$ feet \\
19th & $150$ & $1$ mile & $5$ miles \\
20th & Unlimited & Unlimited & Unlimited \\
\bottomrule
\end{tabular}
\label{table:class-spellbinding-limits}
\caption{Limits on carrying and placing rune stakes at each level.}
\end{table}


\section{Abilities}
\subsection{Place Rune}
At level 1, the spellbinder can place a rune stake at their current location as a bonus action. If the ground at their feet is soft (such as soil or wet cement), they can choose to drive the stake into the ground or lay it flat on the ground. If the ground at their feet is hard (such as rock or tile flooring) or too soft to hold a stake in place (such as dry desert sand), they must lay the stake flat on the ground.

\subsection{Rune Organization}
Each level, the spellbinder creates better systems of keeping their rune stakes organizes, allowing them to carry more rune stakes. See Table \ref{table:class-spellbinding-limits} for the schedule of how many runes stakes can be carried at each level.

\subsection{Fleet of Foot}
At levels 2, 6, 10 and 15, the spellbinder's speed increases by 5 feet.

\subsection{Drop Rune}
At level 2, during a movement action, the spellbinder can drop a rune stake flat on the ground at any point along the path they traveled during their movement. The spellbinder does not need to slow down to do so, and this ability does not use an action or a bonus action beyond that required for movement. The rune stake can be placed at the point the spellbinder started or ended the movement action.

\subsection{Rune Ward}
At level 3, the spellbinder imbues each of their rune stakes with a protective ward that prevents their destruction. All rune stakes crafted after reaching level 3 can only be destroyed by a creature that can cast level 1 spells or higher. Destroying a rune takes one full action.

At level 8, the ward increases in power. Rune stakes created after reaching level 6 can only be destroyed by creatures that can cast level 3 spells.

Any rune stakes created after reaching level 12 can only be destroyed by creatures that can cast level 5 spells.

Any rune stakes created after reaching level 16 can only be destroyed by creatures that can cast level 7 spells.

Any rune stakes created after reaching level 20 can only be destroyed by creatures that can cast level 9 spells.

\subsection{Ability Score Improvement}
When you reach 4th level, and again at 8th, 12th, 16th, and 19th level, you can increase one ability score of your choice by 2, or you can increase two ability scores of your choice by 1. As normal, you can't increase an ability score above 20 using this feature.

\subsection{Throw Rune Stakes}
\label{ability:throw-rune-stakes}
At level 5 when using the Tank class option and level 3 when using the Ranged class option, the spellbinder can use a bonus action to throw a rune stake up to 15 feet. When throwing, the spellbinder makes an Athletics or Acrobatics check (the player decides which). The DC is determined by the DM based on the difficulty of throw; in general, if there are no enemies or obstacles in the way of the throw, it should be rated as Very Easy and have a DC of 5.

If the spellbinder passes the skill check, the rune stake lands where they intended it to land. If the spellbinder fails the skill check, roll two separate $d10$s. Subtract $5$ from the first roll and move the location the rune stick lands that many feet to the right of where the spellbinder was aiming. (If the result is negative, move the rune stick to the left instead.) Subtract $3$ from the second roll and move the location the rune stick lands that many feet closer to the spellbinder. (If the result is negative, move the rune stick further away instead.)\sidenote{The number $3$ and $5$ were chosen so that the rune stake is thrown to the right or left equally as often, but throws usually come up short instead of long.}\sidenote{If the spellbinder fails a skill check on throwing a rune stake 5 or 6 feet, it's possible for the rune stick to land a foot or two behind the spellbinder, opposite the direction it was thrown!}

For example, if the spellbinder fails the skill check and rolls an $8$ for the first $d10$ and a $4$, for the second $d10$, the rune stake lands $8-5=3$ feet to the right of where they were aiming and $4-3=1$ foot closer to the spellbinder than where they were aiming. If the spellbinder rolls $3$ and $1$, the rune stake lands $|3-5|=2$ feet to the left and $|1-3|=2$ feet further away than where they were aiming. If they roll $5$ and $3$, the rune stake happens to land where they were aiming, even though they failed the skill check.

\subsection{Protective Bubble}
At level 13, as an action or reaction, the spellbinder can quickly throw $3$ rune stakes corresponding to the same spell around them in a small triangle and cast that spell. The spellbinder will always be one of the spell's targets, even if the spell has negative effects. If there are any creatures within $5$ feet of the spellbinder, the spellbinder can choose whether they are inside the spell's area of effect.

\subsection{Multicasting}
At level 14, the spellbinder gains the ability to cast multiple spells simultaneously in a single action. All of the rune stakes required to cast the spells must be placed at the time of multicasting. The spellbinder must have spell slots available for all of the spells that are cast, and each spell uses up a spell slot.

The spellbinder chooses the order the spells are cast in. For example, perhaps the spellbinder's party is inside a triangular area of effect for a spell that prevents magical effects, and that triangular area is inside a larger rectangular area of effect for a damage-dealing spell. The spellbinder can activate the protective spell first and the damage-dealing spell second, preventing their party from being hurt by the second spell.

\subsection{Nimble Dodge}
At level 17 when using the Ranged class option, when an attacker that you can see hits you with an attack, you can use your reaction to halve the attack's damage against you.

\subsection{Evasion}
At level 18, you can nimbly dodge out of the way of certain area effects, such as a red dragon's fiery breath or an ice storm spell. When you are subjected
to an effect that allows you to make a Dexterity saving throw to take only half damage, you instead take no damage if you succeed on the saving throw, and only half damage if you fail. This includes being in the area of effect of your own spells!

\subsection{Vastly Telereactive Runes}
At level 19, rune stakes can be placed 1 mile apart from each other when using the Tank class option or 5 miles apart from each other when using the Ranged class option.

\subsection{Infinitely Telereactive Runes}
At level 20, there are no limits on how close rune stakes have to be to each other to case spells. Any of the spellbinder's rune stakes can be used to activate a spell, regardless of their location in the world, as long as they are all in the same plane of existence as each other.

\section{Class Options: Tank Option}
\subsection{Unintimidable}
At levels 3 and 9, increase your maximum hit points by 2.

\subsection{Bonus Proficiencies}
At level 3, gain proficiencies with medium armor, martial melee weapons, simple ranged weapons, and Constitution saving throws.

\subsection{Drop Multiple Runes}
At level 5, when using the Drop Rune ability, the spellbinder can drop as many rune stakes as they wish during a single movement ability as long as each dropped rune stake is at least 5 feet away from the rune stake dropped immediately before it. (The first rune stake, of course, is exempted.) The rune stakes do not have to correspond to the same spell.

\subsection{Tough}
At level 6, gain the Tough feat.

\subsection{Slippery}
At level 14, the spellbinder can pass by an enemy without triggering a reaction.

\section{Class Options: Ranged Option}
\subsection{Throw Runes}
At level 3 (as compared to level 5 for the Tank option), the spellbinder gains the Throw Runes ability described in Subsection \ref{ability:throw-rune-stakes}.

\subsection{Telereactive Runes}
Starting at level 3, the maximum distance between rune stakes increases exponentially each time you level up instead of linearly. See Table \ref{table:class-spellbinding-limits} to see the exact schedule.

\subsection{Throw Runes While Moving}
At level 5, when using the Drop Rune ability, the spellbinder can optionally throw a rune stake instead of dropping it. The spellbinder throws the rune stake up to 15 feet from any point along the path they travel while moving without slowing down. Throwing the rune stake requires the same check that the Throw Rune Stakes ability requires with the same penalty for failure. In general, the DC for a Throwing Runes While Moving check should be higher than the DC would be had the same throw been made using the Throw Runes ability without moving.

\subsection{Mobile}
At level 6, gain the Mobile feat.

\subsection{Rune Orbs}
At level 9, the spellbinder gains the ability to create spherical rune orbs in addition to rune stakes. When crafting a rune stake, the spellbinder decides whether to make it a rune orb instead. Once crafted, a rune stake cannot be turned into or used as a rune orb, and visa versa. Rune orbs count towards the maximum number of rune stakes the spellbinder can carry.

A rune orb is a ball $\frac{1}{4}$ of a foot in diameter. Its volume is $14$ cubic inches ($.0082$ cubic feet), just small enough to fit into a pint glass. Its weight depends on the material it is made out of and is up to the DM.

A rune orb cannot be driven into the ground like a rune stake can. When dropped, it only stays still on level ground or ground, like lumpy grass or soft sand, that doesn't provide a good surface for rolling freely. On smooth, slanted ground, it rolls away in the direction of gravity. When thrown, the orb begins rolling in the direction of the line between the spellbinder and where the orb landed. If it is on level, smooth ground, it continues rolling in that direction until it comes to a stop due to gravity or hits an obstacle, which it bounces off of and continues rolling in a new direction. If it lands on a surface that isn't conducive to rolling, it rolls a short distance and stops. If it lands on a smooth, slanted surface, it begins moving in the direction it was thrown, but quickly changes direction to roll with gravity.

The precise path the orb takes, the speed it moves at, and its location at each turn are too complicated to write full rules for. In the end, the DM chooses the path it travels, how quickly it moves, and where it is located each turn. The spellbinder can describe the path they hope the orb takes when placing or throwing it, and the DM can choose whether to take that intended path into consideration, ask for skill checks to determine how close the actual path is to what was intended, or ignore the intended path altogether. If keeping track of where the orb is each turn is too tedious or involved for your tastes, the DM can decide where its final location is immediately after it is thrown or placed, and the orb remains in that location indefinitely unless it is disturbed.

If an orb is heavy enough, it can be used an an improvised bludgeoning or throwing weapon. The DM decides what the orb's stats are for combat.

\subsection{Hard to See}
At level 14, the spellbinder gains advantage on stealth checks.